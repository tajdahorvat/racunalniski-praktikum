\documentclass{article}
\begin{document}
\section*{Pitagorov izrek}
V pravokotnem trikotniku s katetama \(a\) in \(b\) ter hipotenuzo \(c\) velja
\[ a^2 + b^2 = c^2.\]
Iz tega lahjo izrazimo hipotenuzo \(c\) kot:
\[c = \sqrt{a^2 + b^2}.\]
\end{document}